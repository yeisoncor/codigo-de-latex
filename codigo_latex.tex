\documentclass[12pt,a4paper]{article}
\usepackage[utf8]{inputenc}
\usepackage{amsmath}
\usepackage{amsfonts}
\usepackage{amssymb}
\usepackage{makeidx}
\usepackage{graphicx}
\usepackage[left=2cm,right=2cm,top=2cm,bottom=2cm]{geometry}
\usepackage{anyfontsize}
\usepackage{tikz}
\usepackage{fancybox}
\usepackage{amsmath}
\usepackage{tkz-fct}
\usepackage{subcaption}
\usepackage[spanish]{babel}
\usepackage{xcolor}
\usepackage{float}
\usepackage{pgfplots}
\usetikzlibrary{babel}
\usepackage{pgf}
\usepackage{siunitx}
\usepackage{pst-all}

\begin{document}
\begin{titlepage}
\centering
{\bfseries\LARGE Universidad del Cauca Sede Norte \\}
\vspace{2cm}
{\scshape\Large Facultad De Ingeniería Civil \\}
\vspace{2cm}
{\itshape\Large Trabajo  \\}
\vspace{2cm}
{\scshape\Huge sistemas masa/resorte: Movimiento libre amortiguado \\}

\vspace{2cm}
\vfill
{\Large docente: \\}
{\Large Jhonatan Collazos Ramirez\\}
\vfill
\vfill
{\Large Alumno: \\}
{\Large Yeison Cordoba Gomez \\}
\vfill
\vspace{1cm}
{\Large agosto 2022 \\}
\end{titlepage}

\begin{center}

 \section*{Ecuaciones Diferenciales}

\begin{large}
{Movimiento libre amortiguado}
\end{large}

\end{center}

\begin{flushleft}
\begin{Large}
\textbf{Objetivos:}
\end{Large}
\end{flushleft}

• Investigar, analizar y teorizar los temas requeridos\\
 
• implementar de una excelente manera el programa LaTeX\\

• conocer y explicar el sistemas masa/resorte: Movimiento libre amortiguado.

\vspace{1cm}

\begin{flushleft}
\begin{Large}
\textbf{Marco teórico:}
\end{Large}
\end{flushleft}

\begin{flushleft}

\textbf{• segunda ley de newton:}
define la relación exacta entre fuerza y aceleración matemáticamente. La aceleración de un objeto es directamente proporcional a la suma de todas las fuerzas que actúan sobre él e inversamente proporcional a la masa del objeto, Masa es la cantidad de materia que el objeto tiene.

\vspace{0.5cm}

\textbf{• ley de Hook:}
afirma que la deformación elástica que sufre un cuerpo es proporcional a la fuerza que produce tal deformación, siempre y cuando no se sobrepase el límite de elasticidad.

\vspace{0.5cm}

\textbf{• Sistema masa/resorte:}
consiste en una masa “m” esta va unida a un resorte, que a su vez se halla fijo a una pared.

\vspace{0.5cm}

\textbf{• sistema amortiguado:}
si existe algún elemento en su composición que disipe energía. En un sistema de estas características, la vibración libre que se produce tras alejarlo de su posición de equilibrio y liberarlo a continuación irá disminuyendo a medida que transcurre el tiempo, hasta desaparecer.

\vspace{0.5cm}

\textbf{• LaTeX editor:}
es un sistema de composición de textos, orientado especialmente a la creación de libros, documentos científicos y técnicos que contengan fórmulas matemáticas.

\end{flushleft}

\vspace{1cm}

\section*{Resumen}
A continuación,encotrará información acerca del movimiento libre amortiguado, ramas de aplicación, así como los casos que se pueden dar, además de una breve explicación de donde provienen las fórmulas generales, podrá encontrar gráficas, ejemplos, así como un ejercicio de aplicación para un mayor entendimiento del tema.

\vspace{1cm}

\section*{Introducción}
los sistemas de masa-resorte son muy utilizados en distintas ramas de la ingeniería para moderar vibraciones mecánicas. En la ingeniría civil se utilizan para moderar cimentaciones de maquinaria, muros de retención, análisis de la respuesta dinámica de edificios, entre otros.
\vspace{0.6cm}
El concepto de movimiento armónico libre es un poco irreal, puesto que el movimiento
que describe la ecuación supone que no hay fuerzas retardadoras actuando sobre
la masa en movimiento. A menos que la masa se suspenda en un vacío perfecto, habrá
por lo menos una fuerza de resistencia debida al medio circundante., la masa podría estar suspendida en un medio viscoso o unida a un
dispositivo amortiguador.
 
\vspace{0.6cm}

\section*{Movimiento libre amortiguado}
En el estudio de la mecánica, las fuerzas de amortiguamiento que actúan sobre un cuerpo se consideran proporcionales a una potencia de la velocidad instantánea. En particular, en el análisis posterior se supone que esta fuerza está dada por un múltiplo constante de dx/dt.
Cuando ninguna otra fuerza actúa en el sistema.\\

\vspace{0.6cm}

{\textbf{a partir de la segunda ley de Newton y la ley de Hook obtenemos que:}\\}


\vspace{0.4cm}

donde:

\vspace{0.6cm}

\begin{tabular}{|c|c|}\hline

\textbf{signos:}      & \textbf{significado:}\\ \hline

m              &  corresponde al valor de la masa \\ \hline

$\kappa$  &  corresponde a la constante del resote\\ \hline

$\beta$ &  corresponde a la constante del amortiguador\\ \hline

x'' &  corresponde a la segunda derivada de la posición con respecto al tiempo\\ \hline

x' & corresponde a la primera derivada de la posición con respecto al tiempo\\ \hline

x  &  corresponde a la posición\\ \hline



\end{tabular}

\vspace{0.6cm}

•
${F=-\kappa x -\beta v}$\\

${m*a=-\kappa x -\beta v}$\\


\vspace{0.6cm}

teniendo en cuenta que: \hspace{0.8cm}
$a=\frac{d^2x}{dt^2}$, \hspace{0.8cm} $v={\frac{dx}{dt}}$\hspace{0.6cm} obtenemos:\\
\vspace{0.6cm}

•
{$m{\frac{d^2x}{dt^2}} = -\kappa x -\beta{\frac{dx}{dt}}$}  \hspace{0.8cm}  también lo podemos ver como: \hspace{1cm} $mx'' = -\kappa x - \beta x'$ \\
  
\vspace{0.3cm}

 igualando la ecuación resultante a cero tenemos que:\\
 
 \vspace{0.3cm}
 
• 
$m{\frac{d^2x}{dt^2}} + \beta{\frac{dx}{dt} + \kappa x  =0} $\hspace{0.6cm}  también lo podemos ver como: {\hspace{1cm} $mx'' + \beta x' + \kappa x = 0 $\\}


donde $\beta$ es una constante de amortiguamiento positiva y el signo negativo es una consecuencia del hecho de que la fuerza de amortiguamiento actúa en una dirección
opuesta al movimiento.
Dividiendo la ecuación  entre la masa m, al igual que $\kappa$ es la constante del resorte el cual se encuentra con signo negativo por que se encuentra en dirección opuesta al movimiento, la ecuación diferencial del movimiento libre amortiguado es:


\vspace{0.6cm}
${\frac{d^2x}{dt^2}} + {\frac{\beta}{m}{\frac{dx}{dt}}} +{\frac{\kappa}{m}x} $   \hspace{0.8cm} ó  \hspace{0.8cm} \vspace{0.7cm} •${\frac{d^2x}{dt^2}} + 2\lambda{\frac{dx}{dt} + \omega^{2} x  =0} $\\  



donde: \hspace{0.8cm}
$2\lambda={\frac{\beta}{m}}$, \hspace{0.8cm} $\omega^{2}={\frac{\kappa}{m}}$
\vspace{1cm}


El símbolo $2\lambda$ se usa sólo por conveniencia algebraica, porque la ecuación auxiliar es
$m^{2} + 2\lambda m + \kappa = 0 $ y las raíces correspondientes son:
\vspace{0.6cm}
\begin{center}

$m_{1} = -\lambda + \sqrt{\lambda^{2}-\omega^{2}}$,\hspace{0.8cm}   $m_{2} = -\lambda - \sqrt{\lambda^{2}-\omega^{2}}$
\end{center}

\vspace{1cm}


Ahora se pueden distinguir tres casos posibles dependiendo del signo algebraico, Puesto que cada solución contiene el factor de amortiguamiento, los desplazamientos de la masa se vuelven despreciables conforme el tiempo aumenta.

\section{sobre amortiguado}
En esta situación el sistema está sobre amortiguado porque
el coeficiente de amortiguamiento B es grande comparado con la constante del resorte k.\\ 
 
 \vspace{0.6cm}
Se da cuando: \hspace{0.8cm}   $\lambda^{2}  - \omega^{2} > 0$

 \vspace{0.6cm}
 • la solución general de la ecuación para este caso sería:\\
 
 $x(t) = C_{1}e^{m_{1}t} + C_{2}e^{m_{2}t}$ \hspace{0.8cm} ó \hspace{0.8cm} $x(t) = e^{-\lambda t}(C_{1}e^{\sqrt{\lambda^{2}-\omega^{2}t}} + C_{2}e^{\sqrt{\lambda^{2}-\omega^{2}t}} )$

\begin{figure}[H]

        \centering
        \begin{tikzpicture}
        
        \draw[line width= 0.5, <->](-1,0)--(5.5,0)node      [right]{$x$};
        \draw[line width= 0.5, <->](0,-1)--(0,4.5)node      [above]{$y$};
        \draw[domain= 0:1.8, red, line width= 0.4mm] plot(\x,{\x*\x - 4*\x + 4});
        
        
        \end{tikzpicture}

\end{figure}





\section{críticamente amortiguado}
Este sistema está críticamente amortiguado porque cualquier ligera disminución en la fuerza de amortiguamiento daría como resultado un movimiento oscilatorio.\\

 \vspace{0.6cm}
Se da cuando: \hspace{0.8cm}   $\lambda^{2}  - \omega^{2} = 0$

\vspace{0.6cm}
 • la solución general de la ecuación para este caso sería:\\
 
 
 $x(t) = C_{1}e^{m_{1}t} + C_{2}te^{m_{2}t}$ \hspace{0.8cm} ó \hspace{0.8cm} $x(t) = e^{-\lambda t}(c_{1} + c_{2}t)$
 
 \begin{figure}[h]

        \centering
        \begin{tikzpicture}
        
        \begin{axis}
        [
        xlabel= $x$,
        ylabel= $f(x)$
         ]
         \addplot[
         domain = 1:3,
         samples= 300,
         color= red
         ]         
         {3^(2.5*x-3 -3^(x-1))};
                          
        \end{axis}
        
        \end{tikzpicture}


\end{figure} 






\section{subamortiguado}
En este caso el sistema está subamortiguado puesto que
el coeficiente de amortiguamiento es pequeño comparado con la constante del resorte.\\

 \vspace{0.6cm}
 
Se da cuando: \hspace{0.8cm}   $\lambda^{2}  - \omega^{2} < 0$

\vspace{0.6cm}

las raíces $m_{1}$ y $m_{2}$ ahora son complejas:\\ 

\begin{center}

$m_{1} = -\lambda + \sqrt{\omega^{2}-\lambda^{2}i}$,\hspace{0.8cm}   $m_{2} = -\lambda - \sqrt{\omega^{2}-\lambda^{2}i}$
\end{center}

\vspace{0.6cm}

 • la solución general de la ecuación para este caso sería:\\
 
 $x(t) = e^{-\lambda t}(C_{1}\cos\sqrt{\omega^{2}-\lambda^{2}i} + C_{2}\sin\sqrt{\omega^{2}-\lambda^{2}i}$
 
 \vspace{1cm}
 





\vspace{2cm}
\begin{flushleft}
\begin{center}

\begin{Large}
  \textbf{ a continuacion, procedemos a realizar el ejercicio de aplicación\\}
 \end{Large}
 \end{center}
 
 \vspace{1cm}
 
Una masa de 1 kilogramo se fija a un resorte cuya constante es 16 N/m y luego el sistema completo se sumerge
en un líquido que imparte una fuerza amortiguadora igual
a 10 veces la velocidad instantánea. Determine las ecuaciones de movimiento si:\\ 

\vspace{1cm}

a) al inicio la masa se libera desde un punto situado
a 1 metro abajo de la posición de equilibrio.\\
b) como segunda instancia la masa se libera inicialmente desde un punto situado a 1 metro abajo de la posición de equilibrio con una velocidad ascendente de 12 m/s.

\end{flushleft}

\vspace{0.6cm}

• procedemos a utilizar la formula 

\vspace{0.6cm}

$m{\frac{d^2x}{dt^2}} + \beta{\frac{dx}{dt} + \kappa x  =0} $\\

$mx'' + \beta x' + \kappa x = 0 $

\vspace{0.8cm}

•  remplazamos los valores dados en el problema

\vspace{0.6cm}

\begin{tabular}{|c|c|}\hline

\textbf{datos}     & \textbf{valores}\\ \hline

masa               &   1 \\ \hline

fuerza amortiguadora   &   10\\ \hline

constante del resorte &  16\\ \hline

\end{tabular}

\vspace{0.6cm}

$1x'' + 10x' + 16x = 0 $ 
\vspace{0.8cm}

• la podemos ver como:

\vspace{0.6cm}

$m^{2} + 10m + 16 = 0 $


\vspace{0.8cm}

• aplicando la formula cuadrática tenemos que:

\vspace{0.6cm}

$x=\frac{-b_{-}^{+}\sqrt{b^{2}-4.a.c}}{2.a}$

\vspace{0.8cm}

$m=\frac{-10_{-}^{+}\sqrt{10^{2}-4.1.16}}{2.1}$

$m_{1}= -2$

$m_{2}= -8$

\vspace{1cm}

 utilizaremos la fórmula de solución general para este caso
 
 $x(t) = C_{1}e^{m_{1}t} + C_{2}e^{m_{2}t}$
 
\vspace{1cm}


\textbf{a) para nuestro primer caso remplazamos los valores en nuestra ecuación de movimiento:}

\vspace{0.3cm}

$x(0)=1$, \hspace{1.6cm}  $x'(0)=0$\\

\vspace{0.3cm}

$x(t)=C_{1}e^{-2t} + C_{2}e^{-8t}$  

$x'(t)=-2C_{1}e^{-2t} - 8C_{2}e^{-8t}$ 

\vspace{0.4cm}

remplazamos los valores e igualamos

\vspace{0.3cm}

$x(0): C_{1}e^{0} + C_{2}e^{0}= 1$ \hspace{4.8cm}$x'(0): -2C_{1}e^{0} - 8C_{2}e^{0}= 0$

$x(0): C_{1} + C_{2} = 1$ \hspace{5.5cm}$x'(0): -2C_{1} -8C_{2} = 0 $


\vspace{0.6cm}

resolviendo el sistema de ecuaciones tenemos que:

\vspace{0.3cm}

• $C_{2}=1-C_{1}$   \hspace{5.5cm}   por lo tanto

• $-2C_{1} - 8(1-C_{1})=0 $  \hspace{4.5cm} $C_{2} = 1- C_{1}$

$-2C_{1} - 8 + 8C_{1}=0 $ \hspace{5.2cm} $C_{2} = 1- \frac{4}{3}$

$6C_{1} - 8=0 $  \hspace{6.7 cm}  $C_{2} =- \frac{1}{3}$

$6C_{1} = 8 $

$C_{1} = \frac{4}{3} $\\

•la ecuación del movimiento nos quedaría:

$x(t)= \frac{4}{3}e^{-2t} - \frac{1}{3}e^{-8t}$

\vspace{1.5cm}

\textbf{b) para el segundo caso tenemos como variante la velocidad que corresponde a -12 m/s}

\vspace{0.3cm}

$x(0)=1$, \hspace{1.6cm}  $x'(0)=-12$\\

\vspace{0.4cm}

$x(t)=C_{1}e^{-2t} + C_{2}e^{-8t}$

$x'(t)=-2C_{1}e^{-2t} - 8C_{2}e^{-8t}$

\vspace{0.5cm}

remplazamos los valores e igualamos

\vspace{0.3cm}

$x(0): C_{1}e^{0} + C_{2}e^{0}= 1$ \hspace{4.8cm}$x'(0): -2C_{1}e^{0} - 8C_{2}e^{0}= -12$

$x(0): C_{1} + C_{2} = 1$ \hspace{5.5cm}$x'(0): -2C_{1} -8C_{2} = -12 $


\vspace{0.6cm}

resolviendo el sistema de ecuaciones tenemos que:

\vspace{0.4cm}

• $C_{1} = 1 - C_{2}$   \hspace{5.5cm}   por lo tanto

• $-2C_{1} -8C_{2} = -12$   \hspace{4.5cm} $C_{1} = 1- C_{2}$

$-2(1 - C_{2}) -8C_{2} = -12$   \hspace{3.8cm} $C_{1} = 1- \frac{5}{3}$

$-2 -6C_{2} = -12$   \hspace{5.4 cm}  $C_{1} = \frac{-2}{3}$

$-6C_{2} = -10$

$C_{2} = \frac{5}{3}$


\vspace{0.6cm}

• la ecuación del movimiento para el segundo caso quedaría:

$x(t)=\frac{-2}{3}e^{-2t} + \frac{5}{3}e^{-8t}$

\section*{conclusión}
podemos llegar a la conclusión que en el sistema masa/resorte: movimiento libre amortiguado es un tema no muy complejo el cual tiene gran importancia comprenderlo de una buena manera ya que junto con los demás modelos lineales son importantes puesto que pueden ayudarnos a calcular análisis de la respuesta dinámica de edificios además el  sistema masa/resorte es útil para moderar cimentaciones de maquinaria, muros de retención, entre otros, una de los aspectos más importantes es tener en cuenta la siguiente formula:

\vspace{0.6cm}
\begin{flushleft}
$a{\frac{d^2x}{dt^2}} + b{\frac{dx}{dt} + c x  =0} $\hspace{1.5cm} ó \hspace{1.5cm} $ax'' + bx' + cx = 0$

\vspace{0.6cm}

y además tener un conocimiento básico de la  segunda ley de newton y la ley de Hook que son las bases de donde proviene esta fórmula.  

\vspace{0.7cm}

\section*{Referencias}

\textbf{créditos a los autores de las páginas, sitios web o libros de donde se tomó la información:}

\vspace{0.6cm}

• diferenciales, con problemas con valores en la frontera, séptima edición, autores: Dennis G.Zill, Michael R.Cuellen.2009 disponible en:\\

\vspace{0.3cm}

\emph{https://cutbertblog.files.wordpress.com/2017/10/ecuacionesdiferencialesconproblemasconvaloresenlafrontera7th-141218051407-conversion-gate01.pdf}

\vspace{1.2cm}

• movimiento libre amortiguado, n°23, Dennis G. Zill, plataforma de youtube.2020 disponible en:

\vspace{0.3cm}

\emph{https://youtu.be/iApPMUokZxc}

\vspace{1.2cm}

• ley de Hook, Revisado por Estefania Coluccio Leskow
Ph.D., Departamento de Física, Universidad de Buenos Aires. Última edición: 15 julio, 2021, Disponible en:

\vspace{0.3cm}

\emph{https://concepto.de/ley-de-hooke/}

\vspace{1.2cm}

• segunda ley de newton principio fundamental de la termodinámica, disponible en:


\vspace{0.3cm}

\emph{ $https://imagine.gsfc.nasa.gov/observatories/learning/swift/classroom/docs/law2_guide_spanish.pdf$}

\vspace{1.2cm}

•  resorte/ masa: movimiento libre amortiguado, en diciembre 20, 2021 por Omar González Franco.disponible en:       
                      
\vspace{0.3cm}              

\emph{https://blog.nekomath.com/tag/movimiento-libre-no-amortiguado/} 


\end{flushleft}




\end{document}
